% Bestätigung über Sachzuwendungen, 2014
% Uwe Ziegenhagen, ziegenhagen@gmail.com
% http://code.google.com/p/spendenquittungen-mit-latex/
% 

\documentclass[12pt,ngerman]{scrartcl}
\usepackage[utf8]{inputenc}
\usepackage[T1]{fontenc}
\usepackage{booktabs}
\usepackage{babel}
\usepackage{graphicx}
\usepackage{csquotes}
\usepackage{paralist}
\usepackage{mdframed}
\usepackage{wasysym}
\usepackage{tabu}
\usepackage{ifthen}

\newboolean{sammel} % Sammelbestätigung
\setboolean{sammel}{false}

\renewcommand{\familydefault}{\sfdefault}
\RequirePackage[scaled=0.9]{helvet}
\pagestyle{empty}

\newcommand{\no}{\scalebox{1.5}{\XBox}~} % \CheckedBox
\newcommand{\yes}{\scalebox{1.5}{\Square}}
\usepackage[]{eurosym}

\usepackage{xcolor}
\usepackage[a4paper,left=2cm,right=2cm,top=1cm,bottom=1cm]{geometry}
\setlength{\parindent}{0pt}
\setlength{\parskip}{0pt}


\mdfdefinestyle{MyFormStyle}{%
    linewidth=1pt,
    skipbelow=\topskip,
    skipabove=\topskip
}

\newcommand{\MyForm}[2][1.0cm]{%
    \begin{mdframed}[style=MyFormStyle]%
    {\noindent\footnotesize#2}\vspace{#1}%
    \end{mdframed}%
}

\newcommand{\MyFormBox}[3][1.0cm]{%
    \begin{mdframed}[style=MyFormStyle]%
    {\noindent\footnotesize#2 \vspace*{1em} \par\normalsize #3}\vspace*{#1}%
    \end{mdframed}%
}

\begin{document}
\MyFormBox[0.0cm]{Aussteller (Bezeichnung und Anschrift der steuerbegünstigten Einrichtung)}{Name des Vereins, Anschrift des Vereins,  PLZ und Ort}

{\bfseries\large \ifthenelse{\boolean{sammel}}{Sammelbestätigung}{Bestätigung} über Geldzuwendungen/Mitgliedsbeiträge}\vspace*{1em}

im Sinne des § 10b des Einkommensteuergesetzes an eine der in § 5 Abs. 1 Nr. 9 des Körperschaftsteuergesetzes bezeichneten Körperschaften, Personenvereinigungen oder Vermögensmassen 

\MyFormBox[0.0cm]{Name und Anschrift des Zuwendenden}{<Empfänger der Spendenquittung>}

\begin{tabu}{|[1pt]p{0.3\textwidth}|[1pt]p{0.32\textwidth} |[1pt]p{0.3\textwidth}|[1pt]} \tabucline[1pt]{-}
\scriptsize \ifthenelse{\boolean{sammel}}{Summe}{Betrag} der Zuwendungen - in Ziffern - & \scriptsize- in Buchstaben - & \scriptsize \ifthenelse{\boolean{sammel}}{Zeitraum der Sammelbestätigung}{Tag der Zuwendung} \\ 
\vspace*{1em} & & \\ 
123,45 \euro & --- Einhundertdreiundzwangig --- & 01.01.2001--31.12.2001 \\
\vspace*{1em} & & \\ \tabucline[1pt]{-}
\end{tabu}

\ifthenelse{\boolean{sammel}}{}{\vspace*{0.5em}Es handelt sich um den Verzicht auf Erstattung von Aufwendungen: 	\hspace{1em}	Ja  \yes	\hspace{1em}	Nein  \no}

\vspace*{1.5em}Wir sind wegen Förderung (Angabe des begünstigten Zwecks / der begünstigten Zwecke) \vspace{1em}

% Entweder
\no nach dem letzten uns zugegangenen Freistellungsbescheid bzw. nach der Anlage zum Körperschaftssteuerbescheid des Finanzamts \hspace{5em} StNr \hspace{5em} vom \hspace{5em} nach §~5 Abs. 1 Nr.~9 des Körperschaftssteuergesetzes von der Körperschaftssteuer und nach §~3 Nr.~6 des Gewerbesteuergesetzes von der Gewerbesteuer befreit. \vspace{2em}
 
% oder => siehe den Text hinter \end{document}
 
\begin{mdframed}[style=MyFormStyle]%
Es wird bestätigt, dass die Zuwendung nur zur Förderung der begünstigten Zwecke 1, 2, 3 und 4 AO verwendet wird. 
\end{mdframed} 

\ifthenelse{\boolean{sammel}}{\vspace*{0.5em}Es wird bestätigt, dass über die in der Gesamtsumme enthaltenen Zuwendungen keine weiteren Bestätigungen, weder formelle Zuwendungsbestätigungen noch Beitragsquittungen o.ä., ausgestellt wurden und werden. 

\vspace*{0.5em}Ob es sich um den Verzicht auf Erstattung von Aufwendungen handelt, ist der Anlage zur Sammelbestätigung zu entnehmen.}{}

\vspace*{2.5em} 

Ortsname, den \today \hspace*{20em} Max Mustermann

\hrule

\vspace*{0.5em} (Ort, Datum und Unterschrift des Zuwendungsempfängers) 

\paragraph{Hinweis:} Wer vorsätzlich oder grob fahrlässig eine unrichtige Zuwendungsbestätigung erstellt oder wer veranlasst, dass 
Zuwendungen nicht zu den in der Zuwendungsbestätigung angegebenen steuerbegünstigten Zwecken verwendet 
werden, haftet für die Steuer, die dem Fiskus durch einen etwaigen Abzug der Zuwendungen beim Zuwendenden entgeht (§ 10b Abs. 4 EStG, § 9 Abs. 3 KStG, § 9 Nr. 5 GewStG). 

Diese Bestätigung wird nicht als Nachweis für die steuerliche Berücksichtigung der Zuwendung anerkannt, wenn das Datum des Freistellungsbescheides länger als 5 Jahre bzw. das Datum der vorläufigen Bescheinigung länger als 3 Jahre seit Ausstellung der Bestätigung zurückliegt (BMF vom 15.12.1994 --- BStBl I S. 884). 

\clearpage

{\bfseries\large Anlage zur Sammelbestätigung} \vspace*{2em}

\begin{tabular}{p{0.2\textwidth}p{0.2\textwidth}p{0.3\textwidth}r} \toprule
\bfseries\footnotesize Datum der Zuwendung & \bfseries\footnotesize Art der Zuwendung & \bfseries\footnotesize Verzicht auf die Erstattung von Aufwendungen (ja/nein) & \bfseries\footnotesize Betrag \\ \midrule
01.01.2013 & Mitgliedsbeitrag & nein & 123,00 \euro \\ \midrule[1pt]
Summe: & & & 123,00 \euro \\ \bottomrule[1pt]\bottomrule[1pt]
\end{tabular}


\end{document}

 % für vorläufige Bescheinigung
 \no Wir sind wegen Förderung (Angabe des begünstigten Zwecks / der begünstigten Zwecke) \vspace*{2em} durch vorläufige Bescheinigung des Finanzamts \hspace{5em} StNr \hspace{5em} vom \hspace{5em} ab 
 \hspace{5em} als begünstigten Zwecken dienend anerkannt.


%% Use this if you want it electronically editable PDF.
\usepackage{hyperref}
\newcommand{\MyFormX}[2][1.0cm]{% 
    \begin{mdframed}[style=MyFormStyle]% 
    {\noindent\footnotesize#2}\\% 
    \TextField[format={var f =f.textFont = 'Verdana';f.strokeColor     =['T'];f.fillColor=['T']}, width=\linewidth, height=#1, charsize=10pt]{ }% 
    \end{mdframed}% 
}

\MyFormX{This one is electronically fill-able}
