% Bestätigung über Sachzuwendungen, 2014
% Uwe Ziegenhagen, ziegenhagen@gmail.com
% http://code.google.com/p/spendenquittungen-mit-latex/
% 

\documentclass[12pt,ngerman]{scrartcl}
\usepackage[utf8]{inputenc}
\usepackage[T1]{fontenc}
\usepackage{booktabs}
\usepackage{babel}
\usepackage{graphicx}
\usepackage{csquotes}
\usepackage{paralist}
\usepackage{mdframed}
\usepackage{wasysym}
\usepackage[]{tabu}
\usepackage[]{setspace}
\renewcommand{\familydefault}{\sfdefault}
\RequirePackage[scaled=0.9]{helvet}
\usepackage[]{eurosym}
\pagestyle{empty}

\newcommand{\marked}{\scalebox{1.5}{\XBox}} % \CheckedBox
\newcommand{\notmarked}{\scalebox{1.5}{\Square}}


\usepackage{xcolor}
\usepackage[a4paper,left=2cm,right=2cm,top=1cm,bottom=0.5cm]{geometry}
\setlength{\parindent}{0pt}
\setlength{\parskip}{0pt}


\mdfdefinestyle{MyFormStyle}{%
    linewidth=1pt,
    skipbelow=\topskip,
    skipabove=\topskip
}

\newcommand{\MyForm}[2][1.0cm]{%
    \begin{mdframed}[style=MyFormStyle]%
    {\noindent\footnotesize#2}\vspace{#1}%
    \end{mdframed}%
}

\newcommand{\MyFormBox}[3][1.0cm]{%
    \begin{mdframed}[style=MyFormStyle]%
    {\noindent\footnotesize#2 \vspace*{1em} \par\normalsize #3}\vspace*{#1}%
    \end{mdframed}%
}

\begin{document}
\MyFormBox[0.2cm]{Aussteller (Bezeichnung und Anschrift der steuerbegünstigten Einrichtung)}{Name des Vereins, Anschrift des Vereins, PLZ und Ort}

{\bfseries\large Bestätigung über Sachzuwendungen}\vspace*{1em}
 
im Sinne des § 10b des Einkommensteuergesetzes an eine der in § 5 Abs. 1 Nr. 9 des Körperschaftsteuergesetzes bezeichneten Körperschaften, Personenvereinigungen oder Vermögensmassen 

\MyFormBox[0.2cm]{Name und Anschrift des Zuwendenden}{Adresse des Spenders}

\begin{tabu}{|[1pt]p{0.3\textwidth}|[1pt]p{0.32\textwidth} |[1pt]p{0.3\textwidth}|[1pt]} \tabucline[1pt]{-}
\scriptsize Wert der Zuwendung - in Ziffern - & \scriptsize- in Buchstaben - & \scriptsize Tag der Aufwendungen \\ 
\vspace*{1em} & & \\ 
123,45 \euro & --- Einhundertdreiundzwangig --- &  31.10.2011 \\
\vspace*{1em} & & \\ \tabucline[1pt]{-}
\end{tabu}

\begin{tabu}{|[1pt]p{0.975\textwidth}|} \tabucline[1pt]{-}
\scriptsize Genaue Bezeichnung der Sachzuwendung mit Alter, Zustand, Kaufpreis usw. \\ 
\vspace*{0.5em} \\ 
5 x weiße Farbe, Einzelpreis 40,99 Euro, neu \& original verpackt\\
\vspace*{0.2em} \\ \tabucline[1pt]{-}
\end{tabu}\vspace*{0.5em}

{\footnotesize
\begin{tabular}{cp{0.85\textwidth}}
\hspace{1em} \marked & Die Sachzuwendung stammt nach den Angaben des Zuwendenden aus dem Betriebsvermögen und ist mit dem Entnahmewert (ggf. mit dem niedrigeren gemeinen Wert) bewertet. \\
\hspace{1em} \notmarked & Die Sachzuwendung stammt nach den Angaben des Zuwendenden aus dem Privatvermögen.  \\
\hspace{1em} \notmarked & Der Zuwendende hat trotz Aufforderung keine Angaben zur Herkunft der Sachzuwendung gemacht. \\
\hspace{1em} \notmarked & Geeignete Unterlagen, die zur Wertermittlung gedient haben, z. B. Rechnung, Gutachten, liegen vor. \\
\hspace{1em} \notmarked & Wir sind wegen Förderung (Angabe des begünstigten Zwecks / der begünstigten Zwecke) nach dem letzten uns zugegangenen Freistellungsbescheid bzw. nach der Anlage zum Körperschaftssteuerbescheid des Finanzamts \hspace{5em} StNr \hspace{5em} vom \hspace{5em} nach §~5 Abs. 1 Nr.~9 des Körperschaftssteuergesetzes von der Körperschaftssteuer und nach §~3 Nr.~6 des Gewerbesteuergesetzes von der Gewerbesteuer befreit.
\end{tabular}}

\begin{mdframed}[style=MyFormStyle]%
Es wird bestätigt, dass die Zuwendung nur zur Förderung von Nr.\,1 (Förderung von Wissenschaft und Forschung), Nr.\,5 (Förderung von Kunst und Kultur), Nr.\,7 (Förderung der Erziehung, Volks- und Berufsbildung einschließlich der Studentenhilfe) und Nr.\,13 (Förderung internationaler Gesinnung, der Toleranz auf allen Gebieten der Kultur und des Völkerverständigungsgedankens) des §52 AO verwendet wird.
\end{mdframed}

\vspace*{2.25em}
{\footnotesize Ortname, den \today \hspace*{20em} Max Mustermann}

\hrule

\vspace*{0.5em} (Ort, Datum und Unterschrift des Zuwendungsempfängers) 

{\singlespacing \scriptsize \textbf{Hinweis}: \newline
Wer vorsätzlich oder grob fahrlässig eine unrichtige Zuwendungsbestätigung erstellt oder wer veranlasst, dass Zuwendungen nicht
zu den in der Zuwendungsbestätigung angegebenen steuerbegünstigten Zwecken verwendet werden, haftet für die entgangene Steuer (§ 10b Abs. 4 E StG, § 9 Abs. 3 KStG, § 9 Nr. 5 GewStG). \newline
Diese Bestätigung wird nicht als Nachweis für die steuerliche Berücksichtigung der Zuwendung anerkannt, wenn das Datum des Freistellungsbescheides länger als 5 Jahre bzw. das Datum der vorläufigen Bescheinigung länger als 3 Jahre seit Ausstellung der Bestätigung zurückliegt (BMF vom 15.12.1994 - BStBl I S. 884)}
\end{document}