\documentclass[12pt,ngerman]{scrartcl}
\usepackage[utf8]{inputenc}
\usepackage[T1]{fontenc}
\usepackage{booktabs}
\usepackage{babel}
\usepackage{graphicx}
\usepackage{csquotes}
\usepackage{paralist}
\usepackage{mdframed}
\usepackage{wasysym}
\usepackage[]{tabu}
\usepackage[]{setspace}
\renewcommand{\familydefault}{\sfdefault}
\RequirePackage[scaled=0.9]{helvet}
\usepackage[]{eurosym}
\pagestyle{empty}

\newcommand{\yes}{\scalebox{1.5}{\XBox}} % \CheckedBox
\newcommand{\no}{\scalebox{1.5}{\Square}}


\usepackage{xcolor}
\usepackage[a4paper,left=2cm,right=2cm,top=1cm,bottom=0.5cm]{geometry}
\setlength{\parindent}{0pt}
\setlength{\parskip}{0pt}


\mdfdefinestyle{MyFormStyle}{%
    linewidth=1pt,
    skipbelow=\topskip,
    skipabove=\topskip
}

\newcommand{\MyForm}[2][1.0cm]{%
    \begin{mdframed}[style=MyFormStyle]%
    {\noindent\footnotesize#2}\vspace{#1}%
    \end{mdframed}%
}

\newcommand{\MyFormBox}[3][1.0cm]{%
    \begin{mdframed}[style=MyFormStyle]%
    {\noindent\footnotesize#2 \vspace*{1em} \par\normalsize #3}\vspace*{#1}%
    \end{mdframed}%
}

\begin{document}
\MyFormBox[0.2cm]{Aussteller (Bezeichnung und Anschrift der steuerbegünstigten Einrichtung)}{Name des Vereins \\ Anschrift des Vereins \\ PLZ und Ort}

{\bfseries\large Bestätigung über Sachzuwendungen}\vspace*{1em}
 
im Sinne des § 10b des Einkommensteuergesetzes an eine der in § 5 Abs. 1 Nr. 9 des Körperschaftsteuergesetzes bezeichneten Körperschaften, Personenvereinigungen oder Vermögensmassen 

\MyFormBox[0.2cm]{Name und Anschrift des Zuwendenden}{Adresse des Spenders}

\begin{tabu}{|[1pt]p{0.3\textwidth}|[1pt]p{0.32\textwidth} |[1pt]p{0.3\textwidth}|[1pt]} \tabucline[1pt]{-}
\scriptsize Summe der Zuwendungen - in Ziffern - & \scriptsize- in Buchstaben - & \scriptsize Zeitraum der Aufwendungen \\ 
\vspace*{1em} & & \\ 
123,45 \euro & --- Einhundertdreiundzwangig --- & 01.01.2001--31.12.2001 \\
\vspace*{1em} & & \\ \tabucline[1pt]{-}
\end{tabu}

\begin{tabu}{|[1pt]p{0.975\textwidth}|} \tabucline[1pt]{-}
\scriptsize Genaue Bezeichnung der Sachzuwendung mit Alter, Zustand, Kaufpreis usw. \\ 
\vspace*{0.5em} \\ 
5 x weiße Farbe, Einzelpreis 40,99 Euro, neu \& original verpackt\\
\vspace*{0.2em} \\ \tabucline[1pt]{-}
\end{tabu}\vspace*{0.5em}

{\footnotesize
\begin{tabular}{cp{0.85\textwidth}}
\hspace{1em} \yes & Die Sachzuwendung stammt nach den Angaben des Zuwendenden aus dem Betriebsvermögen und ist mit dem Verkehrswert bewertet.  \\
\hspace{1em} \no & Die Sachzuwendung stammt nach den Angaben des Zuwendenden aus dem Privatvermögen.  \\
\hspace{1em} \no & Der Zuwendende hat trotz Aufforderung keine Angaben zur Herkunft der Sachzuwendung gemacht. \\
\hspace{1em} \yes & Geeignete Unterlagen, die zur Wertermittlung gedient haben, z. B. Rechnung, Gutachten, liegen vor. \\
\hspace{1em} \yes & Wir sind wegen Förderung von §52 AO Nr. 1, 2, 3 und 4 durch vorläufige Bescheinigung des Finanzamtes Ortsname, StNr. 123/4567/8910 vom 01.01.2010 ab 1.02.2010 als steuerbegünstigten Zwecken dienend anerkannt. 
\end{tabular}}

\MyFormBox[0.1cm]{~}%
{
Es wird bestätigt, dass die Zuwendung nur zur Förderung von Nr.\,1 (Förderung von Wissenschaft und Forschung), Nr.\,5 (Förderung von Kunst und Kultur), Nr.\,7 (Förderung der Erziehung, Volks- und Berufsbildung einschließlich der Studentenhilfe) und Nr.\,13 (Förderung internationaler Gesinnung, der Toleranz auf allen Gebieten der Kultur und des Völkerverständigungsgedankens) des §52 AO verwendet wird.
}

\vspace*{2.25em}
{\footnotesize Ortname, den \today \hspace*{20em} Max Mustermann}

\hrule

\vspace*{0.5em} (Ort, Datum und Unterschrift des Zuwendungsempfängers) 

{\singlespacing \scriptsize \textbf{Hinweis}: \newline
Wer vorsätzlich oder grob fahrlässig eine unrichtige Zuwendungsbestätigung erstellt oder wer veranlasst, dass Zuwendungen nicht
zu den in der Zuwendungsbestätigung angegebenen steuerbegünstigten Zwecken verwendet werden, haftet für die entgangene Steuer (§ 10b Abs. 4 E StG, § 9 Abs. 3 KStG, § 9 Nr. 5 GewStG). Diese Bestätigung wird nicht als Nachweis für die steuerliche Berücksichtigung der Zuwendung anerkannt, wenn das Datum des Freistellungsbescheides länger als 5 Jahre bzw. das Datum der vorläufigen Bescheinigung länger als 3 Jahre seit Ausstellung der Bestätigung zurückliegt (BMF vom 15.12.1994 - BStBl I S. 884)}
\end{document}

\hspace{1em} \no & Wir sind wegen Förderung (Angabe des begünstigten Zwecks / der begünstigten Zwecke) xxx nach dem letzten uns zugegangenen Freistellungsbescheid bzw. nach der Anlage zum Körperschaftsteuerbescheid des Finanzamt StNr vom nach § 5 Abs. 1 Nr. 9 des Körperschaftsteuergesetzes von der Körperschaftsteuer und nach § 3 Nr. 6 des Gewerbesteuergesetzes von der Gewerbesteuer befreit. \\



\vspace*{0.5em}Wir sind wegen Förderung (Angabe des begünstigten Zwecks / der begünstigten Zwecke) Nr.1, 5, 7 und 13 AO durch vorläufige Bescheinigung des Finanzamtes Köln-Altstadt, StNr. 214/5853/1069, vom 31.08.2010 ab 31.08.2010 als steuerbegünstigten Zwecken dienend anerkannt. 

\vspace*{0.5em}Es wird bestätigt, dass die Zuwendung nur zur Förderung der begünstigten Zwecke 1, 5, 7 und 13 AO verwendet wird. 

\vspace*{0.5em}Es wird bestätigt, dass über die in der Gesamtsumme enthaltenen Zuwendungen keine weiteren Bestätigungen, weder formelle Zuwendungsbestätigungen noch Beitragsquittungen o.ä., ausgestellt wurden und werden.

\vspace*{2.5em} 



\paragraph{Hinweis:} Wer vorsätzlich oder grob fahrlässig eine unrichtige Zuwendungsbestätigung erstellt oder wer veranlasst, dass 
Zuwendungen nicht zu den in der Zuwendungsbestätigung angegebenen steuerbegünstigten Zwecken verwendet 
werden, haftet für die Steuer, die dem Fiskus durch einen etwaigen Abzug der Zuwendungen beim Zuwendenden entgeht (§ 10b Abs. 4 EStG, § 9 Abs. 3 KStG, § 9 Nr. 5 GewStG). 

Diese Bestätigung wird nicht als Nachweis für die steuerliche Berücksichtigung der Zuwendung anerkannt, wenn das Datum des Freistellungsbescheides länger als 5 Jahre bzw. das Datum der vorläufigen Bescheinigung länger als 3 Jahre seit Ausstellung der Bestätigung zurückliegt (BMF vom 15.12.1994 --- BStBl I S. 884). 

\clearpage

{\bfseries\large Auflistung der einzelnen Geldzuwendungen/Mitgliedsbeiträge:} \vspace*{2em}

\begin{tabular}{p{0.15\textwidth}p{0.12\textwidth}p{0.55\textwidth}} \toprule
\bfseries Datum & \bfseries Betrag & \bfseries Art \\ \midrule
28.04.2013 & 123 & Mitgliedsbeitrag \\ 
28.04.2013 & 123 & Mitgliedsbeitrag \\ 
28.04.2013 & 123 & Mitgliedsbeitrag \\ 
28.04.2013 & 123 & Mitgliedsbeitrag \\ 
28.04.2013 & 123 & Mitgliedsbeitrag \\ 
28.04.2013 & 123 & Mitgliedsbeitrag \\ \midrule
Summe: & 456 & \\ \bottomrule
\end{tabular}



%% Use this if you want it electronically editable PDF.
\usepackage{hyperref}
\newcommand{\MyFormX}[2][1.0cm]{% 
    \begin{mdframed}[style=MyFormStyle]% 
    {\noindent\footnotesize#2}\\% 
    \TextField[format={var f =f.textFont = 'Verdana';f.strokeColor     =['T'];f.fillColor=['T']}, width=\linewidth, height=#1, charsize=10pt]{ }% 
    \end{mdframed}% 
}

\MyFormX{This one is electronically fill-able}
